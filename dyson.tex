\documentclass{amsart}
\usepackage[utf8]{inputenc}

\usepackage[dvipsnames]{xcolor}                 % Package for coloring texts
\usepackage{amsmath}                            % For Maths 
%\usepackage{natbib}
%\bibliographystyle{abbrvnat}

\title{Scribbled Notes}
\author{Sagnik Ghosh}
\date{\\Created: September 14, 2020;\\ Last Updated: \today}

\newtheorem{defn}[]{Definition}
\newtheorem{obs}[]{Observation}
\newtheorem{lem}[]{Lemma}
\newtheorem{thm}[]{Theorem}
\newtheorem{prop}[]{Proposition}
\newtheorem{appr}[]{Approximation}


\begin{document}

\maketitle

\section{Iterating the Dyson Equation}
    We are interested in Phonons. So the theory in consideration is a second order interacting scalar field theory. The Self-Energy ($\Sigma$) has the same causality structure ($D^{-1}$). Hence the Bosonic Dyson Equation can be written in the Keldysh ($\phi^{cl},\phi^{q}$) basis (other indices are suppressed) as follows.
    
    \begin{align}
        \begin{pmatrix}
            D^K & D^R \\
            D^A & 0
        \end{pmatrix}
        =
        \begin{pmatrix}
            D_0^K & D_0^R \\
            D_0^A & 0
        \end{pmatrix} 
        +\begin{pmatrix}
            D_0^K & D_0^R \\
            D_0^A & 0
        \end{pmatrix} 
        \circ\begin{pmatrix}
            0 & \Sigma^A \\
            \Sigma^R & \Sigma^K
        \end{pmatrix}
        \circ\begin{pmatrix}
            D^K & D^R \\
            D^A & 0
        \end{pmatrix} 
    \end{align}
    
    Here the $\circ$ denotes convolution over corresponding time indices. Component wise the equation can be decomposed into,
    
    \begin{align}\label{Retarded}
        D^R(t , t') = D_0^R(t, t') + \int^t_{t'} dt_1\int^{t_1}_{t'} dt_2 D_0^R(t, t_1) \Sigma^R (t_1,t_2) D^R(t_2, t')
    \end{align}
    
    for the \textcolor{red}{retarded} part and,
    
    \begin{multline}\label{Keldysh}
        D^K(t , t') = D_0^R(t, t')  + \int^t_{t'} dt_1\int^{t_1}_{t'} dt_2 D_0^R(t, t_1) \Sigma^R (t_1,t_2) D^K(t_2, t') \\
                                    + \int^t_{t'} dt_1\int^{t_1}_{t'} dt_2 D_0^R(t, t_1) \Sigma^K (t_1,t_2) D^A(t_2, t') 
    \end{multline}
   
   \par for the \textcolor{red}{Keldysh} part. The contribution in eq. (\ref{Keldysh}) from the term $D_0^K\circ \Sigma_0^A\circ D^A$ vanishes due to the vanishing renormalization on $D_0^K$.
    
    
    We look at the retarded component of the Dyson Equation first. The structure for the advanced component is the same and the calculations for the Keldysh component is similar. The Retarded component is given by,

    \begin{align}
        D^R(t , t') = D_0^R(t, t') + \int^t_{t'} dt_1\int^{t_1}_{t'} dt_2 D_0^R(t, t_1) \Sigma^R (t_1,t_2) D^R(t_2, t')
    \end{align}
     The idea is to evolve this to $D^R(t+\epsilon , t')$, and then use this iteratively to generate the dressed Green's Function. The Self Energy ($\Sigma$) and the bare Green's function ($D_0$) are known analytically, hence can be evaluated at any time points. Now,  
     
    \begin{align}\label{Dyson}
        D^R(t+\epsilon , t') = D_0^R(t+\epsilon, t') + \int^{t+\epsilon}_{t'} dt_1\int^{t_1}_{t'} dt_2 D_0^R(t+\epsilon, t_1) \Sigma^R (t_1,t_2) D^R(t_2, t')
    \end{align}
    
    And using the fact that the decomposition of Green's Functions for second order differential operators is given by,
    \begin{align}
        D(t_1 , t_3)=D(t_1 , t_2)\overline{D}(t_2 , t_3)+\overline{D}(t_1 , t_2)D(t_2 , t_3) \; (t_1>t_2>t_3)
    \end{align}
    
    where the bar denotes total derivative w.r.t the first time index eq. (\ref{Dyson}) can be re-written as,
    
    \par 
    \begin{multline}\label{DysonReduced}
        D^R(t+\epsilon , t') = D_0(t+\epsilon , t)\overline{D}(t , t')+\overline{D_0}(t+\epsilon , t)D(t , t')
        \\ + \int^{t+\epsilon}_{t} dt_1\int^{t_1}_{t'} dt_2 D_0^R(t+\epsilon, t_1) \Sigma^R (t_1,t_2) D^R(t_2, t')
    \end{multline}
        
    We can now use a two-point Eulerian Quadrature, to restate the $t_1$ integral,
    
    \begin{multline}\label{Euler}
        D^R(t+\epsilon , t') = D_0(t+\epsilon , t)\overline{D}(t , t')+\overline{D_0}(t+\epsilon , t)D(t , t')
        \\ + \frac{\epsilon}{2} D_0^R(t+\epsilon, t+\epsilon)\int^{t+\epsilon}_{t'} dt_2 \Sigma^R (t+\epsilon,t_2) D^R(t_2, t')
        \\ + \frac{\epsilon}{2} D_0^R(t+\epsilon, t)\int^{t}_{t'} dt_2 \Sigma^R (t,t_2) D^R(t_2, t')
    \end{multline}
    
    or,
    
    \begin{multline}\label{EulerIteration1}
        D^R(t+\epsilon , t') = D_0(t+\epsilon , t)\overline{D}(t , t')+\overline{D_0}(t+\epsilon , t)D(t , t')
        \\ + \frac{\epsilon}{2} D_0^R(t+\epsilon, t)\int^{t}_{t'} dt_2 \Sigma^R (t,t_2) D^R(t_2, t')
    \end{multline}
    
    as the second term drops off, the equal-time retarded Green's Function being zero. Eq (\ref{EulerIteration1}) expresses, the unknown quantity $D^R(t+\epsilon , t')$ in terms of known quantities evaluated at previous time steps and hence can be used to iteratively obtain $D^R$ at all times. Since it does not depend on $D^R(t+\epsilon , t')$, the iteration is not self-consistent.  
    
    The Equation can be further simplified by noticing that,
    
    \begin{align}
        \overline{D}(t , t')=\frac{1}{\epsilon}\big(D(t , t')-D(t-\epsilon , t')\big)
    \end{align}
    
    We define the integral in (\ref{EulerIteration1}) as,
    
    \begin{align}
        I(t,t')=\int^{t}_{t'} dt_2 \Sigma^R (t,t_2) D^R(t_2, t')
    \end{align}
    
    Then using the linerity of integrals and another Eulerian Quadrature we obtain,  
    
    \begin{align}\label{EulerIteration2}
        I(t,t')=I(t-\epsilon,t')+\frac{\epsilon}{2}\big[ \Sigma^R (t,t) D^R(t, t')+\Sigma^R (t,t-\epsilon) D^R(t-\epsilon, t')\big]  
    \end{align}

    This equation saves us a loop. Using (\ref{EulerIteration1}),(\ref{EulerIteration2}), with the following boundary conditions we can obtain $D^R(t,t')$ for all $t$ and a given $t'$.
    
    \begin{align}\label{BoundaryCondition1}
        I(t',t')=0; \; D^R(t',t')=0
    \end{align}

    The Pseudo Code is given by,
    
    \begin{verbatim}
        Import Standard Module;
        Import Math Module;
        
        Begin Program;
        {
        define real t, tprime;
        define real a,b; 		%lower and upper bound on time
        define real tone, ttwo, tthree;	%extra variables
        define real h;			%gap in time, epsilon
        
        define real lambda;		%perturbation parameter
        define real omega, k;		%omega
        
        
        %Module for Euler For the retarded component
        
        DzeroR(t,tprime)= step(t-tprime)*(1/2*omega)sin(omega(t-tprime));
        BarDzeroR(t,tprime)= step(t-tprime)*(1/2)cos(omega(t-tprime));
        SE(t,tprime)=  lambda^2* GzeroK(t,t')(DzeroR(t,tprime)+DzeroA(t,tprime));
        %as of now this formula is dummy
        
        Read value of a;
        Read value of b;
        Read value of h;
        Read value of omega
        Read value of lambda;
        
        For (tprime=a, tprime<=b, tprime=tprime+h)
        	{Set I(a,tprime);
        	Set DR(a,prime);
        	For (t=a, t<=b, t=t+h)
        		{I(t,tprime)=I(t-h,tprime)+h/2*SE(t,t-h)*D(t-h,tprime);
        		 DR(t,tprime)=DzeroR(t,t-h)*1/h*[D(t-h,tprime)-D(t-h,tprime)]
        		        +BarDzero(t,t-h)D(t-h,tprime)+h/2*Dzero(t,t-h)I(t-h,tprime); 
        		 Return DR(t,tprime),I(t,tprime);
        		}
        	}
    \end{verbatim}
    
\section{By-Parts By-Parts Everywhere}

     As is evident in eq (\ref{Euler}), any quadrature that involves end-points will fail to give rise to a self-consistent iteration equation, if applied directly to the $t$ to $t+\epsilon$ integral, since $D^R(t+\epsilon,t+\epsilon)$ goes to zero. And any quadrature that involves mid-points, won't give rise to such terms, in the first place. Hence, construction of self-consistent iteration equation for the second order Green's Functions is slightly convoluted. We attempt to achieve it by applying by-parts to the third piece of eq (\ref{DysonReduced}). The integral can be written as,
     
     \begin{align}
         \int^{t+\epsilon}_{t} dt_1 D_0^R(t+\epsilon, t_1)\int^{t_1}_{t'} dt_2  \Sigma^R (t_1,t_2) D^R(t_2, t')
    \end{align}
    
    Defining,
    
    \begin{align}
              F(t_1,t') = \int^{t_1}_{t'} dt_2  \Sigma^R (t_1,t_2) D^R(t_2, t')
    \end{align}
    
    and treating $D_0^R$ as the first function we can expand the $t_1$ integral by-parts as,
    
        \begin{multline}\label{byparts1}
            \Big[D_0^R(t+\epsilon, t_1)\int^{t_1}_{0}F(t_3,t')dt_3 \Big]^{t+\epsilon}_{t}-\int^{t+\epsilon}_t dt_1 \Big[\frac{d}{dt_1} D_0^R(t+\epsilon,t_1)\int^{t_1}_{t'} F(t_3,t')dt_3\Big]
        \end{multline} 
    
    where the limit on the $t_3$ integral in the last term follows form the Fundamental theorem of Calculus. 
    
    \begin{thm}[First fundamental theorem of Calculus]
        
        Let f be a continuous real-valued function defined on a closed interval [a, b]. Let F be the function defined, for all x in [a, b], by
            \begin{align*}
              F ( x ) = \int_a^x f ( t ) d t   
            \end{align*}
        Then F is uniformly continuous on [a, b] and differentiable on the open interval (a, b), and
            \begin{align*}
                 F' ( x ) = f ( x )
            \end{align*}
        for all x in (a, b). 
    \end{thm}
    
    Only one term survives from the boundary terms, as $D_0^R(t+\epsilon,t+\epsilon)$ goes to zero. And breaking the second integral with an Eulerian Quadrature we obtain,
    
        \begin{multline}\label{byparts2}
            -D_0^R(t+\epsilon, t)\int^{t}_{0}F(t_3,t')dt_3 -\frac{\epsilon}{2} \Big[ \int^{t+\epsilon}_{t'} F(t_3,t')dt_3 \\+\frac{d}{dt} D_0^R(t+\epsilon,t)\int^{t}_{t'} F(t_3,t')dt_3\Big]
        \end{multline} 
        
    Where the fundamental fact exploited is that $\frac{d}{dt}D_0^R(t+\epsilon,t)$ goes to 1 in the limit $t$ tends to equal times, as it behaves as $\theta$ function times cosine. Now the most straight forward way to obtain self consistency would have been to do a quadrature on the second  term.
    
        \begin{multline}\label{byparts3}
            -D_0^R(t+\epsilon, t)\int^{t}_{0}F(t_3,t')dt_3 -\frac{\epsilon}{2} \Big[\frac{d}{dt} \Big(1+D_0^R(t+\epsilon,t)\Big)\int^{t}_{t'} F(t_3,t')dt_3 ]\\ -\frac{\epsilon^2}{4} \Big[ F(t+\epsilon,t')+F(t,t') \Big]
        \end{multline} 
    
    Substituting for $F(t,t')$, 
    
        \begin{multline}\label{byparts4}
            -D_0^R(t+\epsilon, t)\int^{t}_{0}F(t_3,t')dt_3 -\frac{\epsilon}{2} \Big[\frac{d}{dt} \Big(1+D_0^R(t+\epsilon,t)\Big)\int^{t}_{t'} F(t_3,t')dt_3 ]\\ -\frac{\epsilon^2}{4} 
            \Big[ \int^{t+\epsilon}_{t'} dt_2  \Sigma^R (t+\epsilon,t_2) D^R(t_2, t')
            +\int^{t}_{t'} dt_2  \Sigma^R (t,t_2) D^R(t_2, t') \Big]
        \end{multline}
        
    and then doing by-parts on the penultimate terms treating $D^R$ as the first function we obtain,
    
    
        \begin{multline}\label{byparts5}
            -D_0^R(t+\epsilon, t)\int^{t}_{0}F(t_3,t')dt_3 -\frac{\epsilon}{2} \Big[\frac{d}{dt} \Big(1+D_0^R(t+\epsilon,t)\Big)\int^{t}_{t'} F(t_3,t')dt_3 ]\\ -\frac{\epsilon^2}{4} 
            \Big[ \int^{t}_{t'} dt_2  \Sigma^R (t,t_2) D^R(t_2, t')
            \\+  D^R(t+\epsilon, t')\int^{t+\epsilon}_{t'} dt_2  \Sigma^R (t+\epsilon,t_2) -\int^{t+\epsilon}_{t'} dt_2 \big[\frac{d}{dt_2} D^R(t_2 ,t')\int^{t_2}_{t'} F(t_3,t')dt_3\big]
            \Big]
        \end{multline}
        
    The lower limit of the first term goes to zero due to the causality structure of $D^R$. The upper limit gives the self consistent term. Yeet!
    \par Putting everything together the Self-Consistent iteration equation becomes.   
    
    
    
        \begin{multline}\label{selfconsitentdyson1}
            \mathbf{D^R(t+\epsilon , t')} = D_0(t+\epsilon , t)\overline{D}(t , t')+\overline{D_0}(t+\epsilon , t)D(t , t')
        \\  -D_0^R(t+\epsilon, t)\int^{t}_{0}F(t_3,t')dt_3 -\frac{\epsilon}{2} \Big[\frac{d}{dt} \Big(1+D_0^R(t+\epsilon,t)\Big)\int^{t}_{t'} F(t_3,t')dt_3 ]\\ -\frac{\epsilon^2}{4} 
            \Big[ \int^{t}_{t'} dt_2  \Sigma^R (t,t_2) D^R(t_2, t')
            -\int^{t+\epsilon}_{t'} dt_2 \big[\frac{d}{dt_2} D^R(t_2 ,t')\int^{t_2}_{t'} F(t_3,t')dt_3\big]
            \\+  \mathbf{D^R(t+\epsilon, t')}\int^{t+\epsilon}_{t'} dt_2  \Sigma^R (t+\epsilon,t_2)\Big]
        \end{multline}
        
    The bad thing about eq(\ref{selfconsitentdyson1}) is that there are painful double integrals, which are costly for computation. But the good thing is they can be individually simplified using something similar to (\ref{EulerIteration1}) and then breaking the increment using quadrature.
    We can not really use Euler everywhere, as Euler has second order error, and that will add up. We can use something like Simpson instead at some place, if necessary. But the basic idea remains the same.    
    The pseudo-code is similar to the non self consistent case. The idea is to run a Newton-Rhapson at every step in place of the simple iteration, to find an implicit expression from \ref{selfconsitentdyson1}.
    
\end{document}
